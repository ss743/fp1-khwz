\documentclass[12pt]{article}

\usepackage{fancyhdr}
\usepackage{geometry}
\usepackage{ucs}
\usepackage[utf8x]{inputenc}
\usepackage[T1]{fontenc}
\usepackage[ngerman]{babel}
\usepackage{amsmath,amssymb,amstext}
\usepackage{hyperref}
\usepackage{cancel}
\usepackage{dsfont}
\usepackage{physics}
\usepackage{lmodern}
\usepackage{enumerate}
\usepackage{enumitem}
\usepackage{graphicx}
\usepackage{listings, color}
\usepackage[labelfont=bf]{caption}
\usepackage{titling}

\lstset{basicstyle=\scriptsize} %Quellcode mit Umlauten und ganz klein
\lstset{literate=
  {Ö}{{\"O}}1
  {Ä}{{\"A}}1
  {Ü}{{\"U}}1
  {ß}{{\ss}}2
  {ü}{{\"u}}1
  {ä}{{\"a}}1
  {ö}{{\"o}}1
}


%Geometrie----------------------------------------------------------------------------------------------------------

\geometry{a4paper, top=25mm, left=15mm, right=15mm, bottom=25mm,headsep=10mm, footskip=10mm}
\pagestyle{fancy}
\setlength{\parindent}{0pt} %Zeileneinrückung

\fancyhf{} %Setzt voreingestellte Kopf-und Fußzeilen-Eigenschaften zurück

\lhead{\nouppercase{\leftmark}}
\chead{}
\rhead{\thepage}

\lfoot{}
\cfoot{}
\rfoot{}

\title{\vspace{0cm}{\Huge Fortgeschrittenen-Praktikum I:\\ \vspace{1cm} Kurze Halbwertszeiten}}
\author{Saskia Bondza\\Simon Stephan}
\date{Durchgeführt am 02.09.2016 und 05.09.2016}

\pretitle{%
  \begin{center}
  \LARGE
  \includegraphics[width=6cm,]{figures/siegel}\\[\bigskipamount]
}
\posttitle{\end{center}}

%neue Commands----------------------------------------------------------------------------------------------------------
\newcommand{\nab}{\vec{\nabla}} %direkter Befehl mit Vektorpfeil
\newcommand{\gra}[2]{
	\begin{minipage}{\textwidth}
		\centering
		\includegraphics[width=0.7\textwidth]{figures/#1.png}
		\captionof{figure}{#2}
	\end{minipage}
	}


%Titel,Inhalt----------------------------------------------------------------------------------------------------------

\begin{document}
\pagenumbering{gobble} %verstecke Seitenzahl
\maketitle
\newpage

\thispagestyle{empty}
\tableofcontents
\newpage

%Schreiben----------------------------------------------------------------------------------------------------------
\pagenumbering{arabic} %verstecke Seitenzahl
\section{Einleitung}


In diesem Versuch soll die Halbwertszeit des $14,4$ keV-Zustands von $^{57}$Fe bestimmt werden. Dieser Zustand kann durch den Zerfall von $^{57}$Co durch Elektroneneinfang und anschließenden $\gamma$-Zerfall des $^{57}$Fe erzeugt werden. Zum Messen der Halbwertszeit dieses Zerfalls wird die Zeit zwischen der Detektierung des $122,1$ keV-Photons, welches beim Entstehen des $14,4$ keV-Zustands ausgesendet wird, und der Detektierung des $14,4$ keV-Photons, welches beim Zerfall des $14,4$ keV-Zustands in den Grundzustand ensteht, gemessen. Da nur ein kleiner Teil der mit $14,4$ keV angeregten Eisenkerne unter Aussendung eines Photons zerfällt und so nicht zu jedem $122,1$ keV-Photon ein $14,4$ keV-Photon entsteht, wird zum Messen die Methode der verzögerten Koinzidenzen benutzt.




\newpage
\section[Aufgabenstellung]{Aufgabenstellung}%$^{\cite{anleitung}}$}
Zum Bestimmen der Halbwertszeit werden zunächst die Energie-Spektren von $^57$Co und $241$Am aufgenommen. Anschließend wird eine Zeiteichung des Time to Amplitude Converters (TAC) durchgeführt. Dann wird eine Untergrundmessung ohne Signalverzögerung durchgeführt um die zufälligen Koinzidenzen zu messen. Nun werden die verzögerten Koinzidenzen gemessen und daraus die Halbwertszeit des $14,4$ keV-Zustands von $^{57}$Fe bestimmt.





\newpage
\section{Theoretische Grundlagen}
\subsection{Radioaktive Zerfälle}
Instabile Atomkerne gehen, je nach Art des Zerfalls, unter Aussendung von ionisierender Strahlung und Aussendung von Teilchen spontan in einen anderen Atomkern über. Im Folgenden werden die verschiedenen Arten radioaktiver Zerfälle erläutert.
\subsubsection{$\alpha$-Zerfall}
Beim $\alpha$-Zerfall geht ein schwerer, instabiler Kern unter Aussendung eines $\alpha$-Teilchens, einem Heliumkern (zwei Protonen und zwei Neutronen) in einen stabilen Kern über. Die Massenzahl des neuen Atomkerns ist dabei um vier gesunken, die Ordnungszahl verringert sich um zwei.
\subsubsection{$\beta$-Zerfall}
Der $\beta$-Zerfall wird durch verschiedene Prozesse der schwachen Wechselwirkung beschrieben bei denen Elektronen  und Positronen auftreten. Bei diesem Vorgang wird durch W-Boson-Austausch ein Proton in ein Neutron umgewandelt bzw. ein Neutron in ein Proton.

\paragraph*{$\beta^-$-Zerfall:} 
Bei dieser Art des Zerfalls wandelt sich ein Neutron im Kern unter Aussendung eines Elektrons und eines Elektron-Antineutrinos in ein Proton um. 
\begin{align*}
n &\rightarrow p + e^- + \bar{\nu_e}\\
{}_Z^A X &\rightarrow {}_{Z+1}^A Y + e^- + \bar{\nu_e}
\end{align*}

Hierbei entspricht X dem Mutternuklid und Y dem Tochternuklid. Die Ordnungszahl erhöht sich bei diesem Vorgang um eine Einheit, die Massenzahl ändert sich nicht.

\paragraph*{$\beta^+$-Zerfall:}
Der $\beta^+$-Zerfall ist dem $\beta^-$-Zerfall sehr ähnlich, Hier wandelt sich ein Proton im Kern unter Aussendung eines Positrons und eines Elektron-Neutrinos in ein Neutron um.
\begin{align*}
p &\rightarrow n + e^+ + \nu_e\\
{}_Z^A X &\rightarrow {}_{Z-1}^A Y + e^+ + \nu_e
\end{align*}

Bei diesem Prozess sinkt die Kernladungszahl um eine EInhiet während die Massenzahl wie auch beim $\beta^-$-Zerfall unverändert bleibt.
\subsubsection{Elektroneneinfang}
\subsubsection{$\gamma$-Zerfall}
\subsubsection{Innere Konversion}


\newpage
\section{Versuchsaufbau}


\newpage
\section{Versuchsdurchführung}



\newpage
\section{Auswertung}


\newpage
\section{Zusammenfassung/Diskussion}


\newpage
\section{Anhang}

\subsection{Tabellen}

%\subsubsection{$\alpha$-Plateau Samarium}
%\lstinputlisting[language=MATLAB]{Rohdaten/alphaPlateau_Sm.txt}


%\newpage
%\subsection{Quellcode (MATLAB)}
%\lstinputlisting[language=MATLAB]{Rohdaten/alpha.m}

\newpage
\subsection{Laborheft}
%\begin{minipage}{\textwidth}
%\centering
%\includegraphics[width=0.9\textwidth]{figures/IMG_20151002_141014.jpg}
%\end{minipage}

\newpage
\listoffigures

%Literatur----------------------------------------------------------------------------------------------------------

%\cite{les}
\newpage
\thispagestyle{empty}
\begin{thebibliography}{9}

%\bibitem{staat}
%  Tobijas Kotyk,
%  \emph{Versuche zur Radioaktivität im Physikalischen Fortgeschrittenen Praktikum an der Albert-Ludwigs-Universität Freiburg},
%  Albert-Ludwigs-Universität, Freiburg,
%  2005
  

  
%\bibitem{molmasse}
%  \emph{http://www.convertunits.com/molarmass/<ELEMENTNAME AUF ENGLISCH>}, Stand 28.09.2015
  

\end{thebibliography}

\end{document}