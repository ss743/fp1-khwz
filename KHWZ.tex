\documentclass[12pt]{article}

\usepackage{fancyhdr}
\usepackage{geometry}
\usepackage{ucs}
\usepackage[utf8x]{inputenc}
\usepackage[T1]{fontenc}
\usepackage[ngerman]{babel}
\usepackage{amsmath,amssymb,amstext}
\usepackage{hyperref}
\usepackage{cancel}
\usepackage{dsfont}
\usepackage{physics}
\usepackage{lmodern}
\usepackage{enumerate}
\usepackage{enumitem}
\usepackage{graphicx}
\usepackage{listings, color}
\usepackage[labelfont=bf]{caption}
\usepackage{titling}

\lstset{basicstyle=\scriptsize} %Quellcode mit Umlauten und ganz klein
\lstset{literate=
  {Ö}{{\"O}}1
  {Ä}{{\"A}}1
  {Ü}{{\"U}}1
  {ß}{{\ss}}2
  {ü}{{\"u}}1
  {ä}{{\"a}}1
  {ö}{{\"o}}1
}


%Geometrie----------------------------------------------------------------------------------------------------------

\geometry{a4paper, top=25mm, left=15mm, right=15mm, bottom=25mm,headsep=10mm, footskip=10mm}
\pagestyle{fancy}
\setlength{\parindent}{0pt} %Zeileneinrückung

\fancyhf{} %Setzt voreingestellte Kopf-und Fußzeilen-Eigenschaften zurück

\lhead{\nouppercase{\leftmark}}
\chead{}
\rhead{\thepage}

\lfoot{}
\cfoot{}
\rfoot{}

\title{\vspace{0cm}{\Huge Fortgeschrittenen-Praktikum I:\\ \vspace{1cm} Kurze Halbwertszeiten}}
\author{Saskia Bondza\\Simon Stephan}
\date{Durchgeführt am 02.09.2016 und 05.09.2016}

\pretitle{%
  \begin{center}
  \LARGE
  \includegraphics[width=6cm,]{figures/siegel}\\[\bigskipamount]
}
\posttitle{\end{center}}

%neue Commands----------------------------------------------------------------------------------------------------------
\newcommand{\nab}{\vec{\nabla}} %direkter Befehl mit Vektorpfeil
\newcommand{\gra}[2]{
	\begin{minipage}{\textwidth}
		\centering
		\includegraphics[width=0.7\textwidth]{figures/#1.png}
		\captionof{figure}{#2}
	\end{minipage}
	}


%Titel,Inhalt----------------------------------------------------------------------------------------------------------

\begin{document}
\pagenumbering{gobble} %verstecke Seitenzahl
\maketitle
\newpage

\thispagestyle{empty}
\tableofcontents
\newpage

%Schreiben----------------------------------------------------------------------------------------------------------
\pagenumbering{arabic} %verstecke Seitenzahl
\section{Einleitung}


In diesem Versuch soll die Halbwertszeit des $14,4$ keV-Zustands von $^{57}$Fe bestimmt werden. Dieser Zustand kann durch den Zerfall von $^{57}$Co durch Elektroneneinfang und anschließenden $\gamma$-Zerfall des $^{57}$Fe erzeugt werden. Zum Messen der Halbwertszeit dieses Zerfalls wird die Zeit zwischen der Detektierung des $122,1$ keV-Photons, welches beim Entstehen des $14,4$ keV-Zustands ausgesendet wird, und der Detektierung des $14,4$ keV-Photons, welches beim Zerfall des $14,4$ keV-Zustands in den Grundzustand ensteht, gemessen. Da nur ein kleiner Teil der mit $14,4$ keV angeregten Eisenkerne unter Aussendung eines Photons zerfällt und so nicht zu jedem $122,1$ keV-Photon ein $14,4$ keV-Photon entsteht, wird zum Messen die Methode der verzögerten Koinzidenzen benutzt.




\newpage
\section[Aufgabenstellung]{Aufgabenstellung}%$^{\cite{anleitung}}$}
Zum Bestimmen der Halbwertszeit werden zunächst die Energie-Spektren von $^57$Co und $^241$Am aufgenommen. Anschließend wird eine Zeiteichung des Time to Amplitude Converters (TAC) durchgeführt. Dann wird eine Untergrundmessung ohne Signalverzögerung durchgeführt um die zufälligen Koinzidenzen zu messen. Nun werden die verzögerten Koinzidenzen gemessen und daraus die Halbwertszeit des $14,4$ keV-Zustands von $^{57}$Fe bestimmt.





\newpage
\section{Theoretische Grundlagen}
\subsection{Zerfallsgesetz}
Der radioaktive Zerfall ist ein statistischer Prozess d.h. der genaue Zeitpunkt zu dem ein Kern zerfällt kann nicht vorhergesagt werden, stattdessen wird mit Wahrscheinlichkeiten gearbeitet. Die Zerfallsrate $dN/dt$ ist abhängig von der momentanen ANzahl $N$ der Atome.
\[\frac{dN}{dt}=-\lambda N\]
wobei $\lambda$ die Zerfallskonstante ist, welche den Bruchteil der zu Beginn vorhandenen Kerne, die pro Zeiteinheit zerfallen beschreibt. Zerfallskonstanten einzelner Zerfälle addieren sich, zerfällt ein Kern auf mehrere Arten. Durch Lösen dieser Differentialgleichung erhält man die Anzahl der Kerne $N$ zum Zeitpunkt $t$.
\[N(t)=N_0e^{-\lambda t}\hspace{0.5cm}\]
wobei $N_0=N(t=0)$ die anfängliche Anzahl der Kerne beschreibt. Hieraus lässt sich dann die mittlere Lebensdauer $\tau$ bestimmen:
\[\tau=\frac{1}{\lambda}\]
welche mit der Halbwertszeit $T_{1/2}$ wie folgt zusammenhängt:
\[T_{1/2} = ln2\cdot \tau = \frac{ln2}{\lambda}\]
Eine weitere wichtige Größe ist die Aktivität $A$ welche die Anzahl der zerfälle pro Sekunde beschreibt und in Becquerel ( $Bq=1/s$) gemeesen wird:
\[A = \lambda N = \frac{N}{\tau} = \frac{ln2 N}{T_{1/2}}\]
\subsection{Radioaktive Zerfälle}
Instabile Atomkerne gehen, je nach Art des Zerfalls, unter Aussendung von ionisierender Strahlung und Aussendung von Teilchen spontan in einen anderen Atomkern über. Im Folgenden werden die verschiedenen Arten radioaktiver Zerfälle erläutert.
\subsubsection{$\alpha$-Zerfall}
Beim $\alpha$-Zerfall geht ein schwerer, instabiler Kern unter Aussendung eines $\alpha$-Teilchens, einem Heliumkern (zwei Protonen und zwei Neutronen) in einen stabilen Kern über. Die Massenzahl des neuen Atomkerns ist dabei um vier gesunken, die Ordnungszahl verringert sich um zwei.
\subsubsection{$\beta$-Zerfall}
Der $\beta$-Zerfall wird durch verschiedene Prozesse der schwachen Wechselwirkung beschrieben bei denen Elektronen  und Positronen auftreten. Bei diesem Vorgang wird durch W-Boson-Austausch ein Proton in ein Neutron umgewandelt bzw. ein Neutron in ein Proton.

\paragraph*{$\beta^-$-Zerfall:} 
Bei dieser Art des Zerfalls wandelt sich ein Neutron im Kern unter Aussendung eines Elektrons und eines Elektron-Antineutrinos in ein Proton um. 
\begin{align*}
n &\rightarrow p + e^- + \bar{\nu_e}\\
{}_Z^A X &\rightarrow {}_{Z+1}^A Y + e^- + \bar{\nu_e}
\end{align*}

Hierbei entspricht X dem Mutternuklid und Y dem Tochternuklid. Die Ordnungszahl erhöht sich bei diesem Vorgang um eine Einheit, die Massenzahl ändert sich nicht.

\paragraph*{$\beta^+$-Zerfall:}
Der $\beta^+$-Zerfall ist dem $\beta^-$-Zerfall sehr ähnlich, Hier wandelt sich ein Proton im Kern unter Aussendung eines Positrons und eines Elektron-Neutrinos in ein Neutron um.
\begin{align*}
p &\rightarrow n + e^+ + \nu_e\\
{}_Z^A X &\rightarrow {}_{Z-1}^A Y + e^+ + \nu_e
\end{align*}

Bei diesem Prozess sinkt die Kernladungszahl um eine Einheit während die Massenzahl wie auch beim $\beta^-$-Zerfall unverändert bleibt.
\subsubsection{Elektroneneinfang}
Die meisten schweren, instabilen Kerne mit Protonenüberschuss wandeln sich durch Elektroneneinfang in stabile Kerne um.
Der Elektroneneinfang führt effektiv im Kern zum gleichen Resultat wie der $\beta^+$-Zerfall, d.h. die Ordnungszahl verringert sich um eine Einheit, die Massenzahl bleibt unverändert. Meist wird aus der Kern-nächsten Schale, der K-Schale, ein Bahnelektron unter Aussendung eines Neutrinos eingefangen. Die Lücke, die hierbei entsteht, wird meist durch ein aus der L-Schale stammendes Elektron unter Emission von Röntgenstrahlung oder Auger-Elektronen (s.u.) gefüllt. Dieser Vorgang wiederholt sich mit den weiter außen liegenden Schalen.
\[{}_Z^A X + e^- \rightarrow\ {}_{Z-1}^A Y + \nu_e\]
\subsubsection{$\gamma$-Zerfall}

Der $\gamma$-Zerfall ist oft eine Begleiterscheinung der bereits diskutierten Kernzerfälle: Befindet sich das Tochternuklid nach dem Zerfall in einem angeregten Zustand, geht es über Aussendung eines $\gamma$-Quants (Photon)  in einen energetisch niedrigeren Zustand über. Die Energie dieser Photonen liegt dabei im Bereich von keV bis MeV. In Materie fällt die Intensität$I_d$ der $\gamma$-Strahlung exponentiell mit der Strecke $d$ ab:
\[I_d=e^{\mu d}\]
Dabei ist $\mu$ der Absorptionskoeffizent des Materials und $I_0$ die Anfangsintensität.

\subsubsection{Innere Konversion}
Konkurrierend zum $\gamma$-Zerfall gibt es auch die Innere Konversion. Bei dieser gehen angeregte Kernzustände strahlungslos in den Grundzustand über, wobei die Energie an ein Elektron übertragen und dieses abgestrahlt wird. Die so entstandene Lücke in der Atomhülle wird dann unter Emission von Röntgenstrahlung oder Auger-Elektronen (s.u.) gefüllt.

\subsection{Prozesse in der Atomhülle}
Bei manchen Zerfallsprozessen entsteht in der Atomhülle durch Emission eines Elektrons ein Loch. Dieses Loch wird gefüllt, indem ein Elektron einer höheren Schale in die Schale des Loches wechselt und dabei Energie freigibt. Diese Energie kann entweder über direkte Emission eines Photons (Röntgenstrahlung) oder über die Emission eines anderen Elektrons (Auger-Elektron) abgegeben werden. Dadurch entstehen eine oder zwei neue Lücken, welche wiederum über die weitere Emission von Röntgenstrahlung oder eines Auger-Elektrons geschlossen werden können.









\subsection{Szintillationszähler}






\subsection{Wechselwirkung von $\gamma$-Strahlung mit Materie}
Um i$\gamma$-Strahlung detektieren zu können, muss die $\gamma$-Strahlung mit Materie wechselwirken. Wir unterscheiden dabei grundsätzlich drei Prozesse: Den Photoeffekt, den Compton-Effekt und die Paarbildung.
 \subsubsection{Photo-Effekt}
 Beim Photoeffekt dringt ein Photon in das Atom ein und überträgt seine gesamte Energie an ein Elektron der inneren Schalen. Dabei wird Energie auf dieses Elektron übertragen, es wird aus der Atomhülle befreit und erhält kinetische Energie. Die hier entstandene Lücke wird über Abstrahlung eines $\gamma$-Quants oder eines Auger-Elektrons wieder gefüllt.
 Der Photoeffekt findet typischerweise bei Energien bis zu 200 eV statt.
 \subsubsection{Compton-Effekt}
 Beim Compton-Effekt trifft ein einfallendes $\gamma$-Quant auf ein freies oder nur leicht gebundenes Elektron und überträgt einen Teil seiner Energie auf dieses. Dieser Prozess findet mesit bei Energien zwischen 200keV und 5MeV statt.
 \subsubsection{Paarbildung}
 Bei der Paarbildung entsteht durch die Wechselwirkung des $\gamma$-Quants mit dem elektromagnetischen Feld des Atomkerns oder eines Elektrons ein Teilchen-Antiteilchen-Paar i.e. Elektronen-Positronen-Paar.Paarbildung ist für Energien über 1,022 MeV möglich. Die über diesen Grenzwert hinausgehende Energie wird auf die entstandenen Teilchen übertragen, der Imuls wird vom Kern aufgenommen. Da das Positron nicht lange alleine existieren kann, vereinigt es sich unter Abstrahlung von zwei $\gamma$-Quanten mit einem Elektron.
 \subsection{Messprinzip}
 \subsubsection{Messung der Zerfallsdauern}
 Um die Halbwertszeit des $14,4$ keV-Zustands von $^{57}$Fe zu bestimmen, messen wir die einzelnen Lebensdauern der angeregten $^{57}$Fe-Atome. Der Zustand entsteht unter Emission eines $122,1$ keV-$\gamma$-Quants und zerfällt unter Emission eines $14,4$ keV-$\gamma$-Quants, so dass wir die Zeitmessung mit Detektion des $122,1$ keV-Photons starten und mit Detektion des $14,4$ keV-Photons beenden könnten. Bei diesem Verfahren allerdings tritt das Problem auf, dass nur ca. 10\% der $^{57}$Fe-Atome über $\gamma$-Zerfall und ca. 90\% über innere Konversion zerfallen. Wenn nun ein $122,1$ keV-Photon detektiert wird und kein dazugehöriges $14,4$ keV-Photon erzeugt wird, dann wird die Messung nicht beendet, wodurch die Totzeit vergrößert und die Messung verfälscht wird.
 Um dieses Problem zu umgehen, benutzt man die Methode der verzögerten Koinzidenzen. Bei dieser Methode wird das Signal der $122,1$ keV-Photonen verzögert, so dass das Signal nach dem Signal der $14,4$ keV-Photonen ankommt. Dann wird das Signal der $14,4$ keV-Photonen als Start- und das der $122,1$ keV-Photonen als Stop-Signal verwendet. Die gemessene Differenz wird nun von der Verzögerung abgezogen, um die tatsächliche Lebensdauer zu erhalten.
 Der radioaktive Zerfall ist ein statistischer Prozess (siehe <REFERENZ>), mit der Anzahl $N(t)$ der nach einer Zeit $t$ noch vorhandenen Kerne: $N(t)=N_0\exp({-\frac t \tau})$ Die gemessenen Zeiten $\Delta t$ folgen ebenso dieser Exponentialverteilung um die mittlere Lebensdauer $\tau$.
 \subsubsection{Nachweis der $\gamma$-Strahlung}
 Die emittierten Photonen müssen nun detektiert und ihre Energie bestimmt werden. Dazu wird ein Energie-sensitiver Detektor, eine Kombination aus Szintillator und Photomultiplier, verwendet.
 \paragraph{{Szintillator}} Ein Szintillator detektiert Teilchen eines bestimmten Energiebereichs. Es gibt organische und anorganische Szintillatoren, wobei in diesem Versuch anorganische NaI(Tl)-Szintillatoren verwendet werden. Dieser besteht aus einem mit Thalium dotierten NaI-Kristall, in welchem die mit entsprechender Energie eintreffenden Photonen ihre Energie durch den Photo- oder Compton-Effekt an Elektronen abgeben. Diese Elektronen werden nun angeregt und später unter Emission  niederenergetischer Photonen wieder abgeregt. Die Dotierung mit Thalium verhindert, dass die emittierten Elektronen wieder absorbiert werden.
 \paragraph{Photomultiplier}
\newpage
\section{Versuchsaufbau}


\newpage
\section{Versuchsdurchführung}



\newpage
\section{Auswertung}


\newpage
\section{Zusammenfassung/Diskussion}


\newpage
\section{Anhang}

\subsection{Tabellen}

%\subsubsection{$\alpha$-Plateau Samarium}
%\lstinputlisting[language=MATLAB]{Rohdaten/alphaPlateau_Sm.txt}


%\newpage
%\subsection{Quellcode (MATLAB)}
%\lstinputlisting[language=MATLAB]{Rohdaten/alpha.m}

\newpage
\subsection{Laborheft}
%\begin{minipage}{\textwidth}
%\centering
%\includegraphics[width=0.9\textwidth]{figures/IMG_20151002_141014.jpg}
%\end{minipage}

\newpage
\listoffigures

%Literatur----------------------------------------------------------------------------------------------------------

%\cite{les}
\newpage
\thispagestyle{empty}
\begin{thebibliography}{9}

%\bibitem{staat}
%  Tobijas Kotyk,
%  \emph{Versuche zur Radioaktivität im Physikalischen Fortgeschrittenen Praktikum an der Albert-Ludwigs-Universität Freiburg},
%  Albert-Ludwigs-Universität, Freiburg,
%  2005
  

  
%\bibitem{molmasse}
%  \emph{http://www.convertunits.com/molarmass/<ELEMENTNAME AUF ENGLISCH>}, Stand 28.09.2015
  

\end{thebibliography}

\end{document}