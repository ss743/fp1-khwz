\documentclass[12pt]{article}

\usepackage{fancyhdr}
\usepackage{geometry}
\usepackage{ucs}
\usepackage[utf8x]{inputenc}
\usepackage[T1]{fontenc}
\usepackage[ngerman]{babel}
\usepackage{amsmath,amssymb,amstext}
\usepackage{hyperref}
\usepackage{cancel}
\usepackage{dsfont}
\usepackage{physics}
\usepackage{lmodern}
\usepackage{enumerate}
\usepackage{enumitem}
\usepackage{graphicx}
\usepackage{listings, color}
\usepackage[labelfont=bf]{caption}
\usepackage{titling}

\lstset{basicstyle=\scriptsize} %Quellcode mit Umlauten und ganz klein
\lstset{literate=
  {Ö}{{\"O}}1
  {Ä}{{\"A}}1
  {Ü}{{\"U}}1
  {ß}{{\ss}}2
  {ü}{{\"u}}1
  {ä}{{\"a}}1
  {ö}{{\"o}}1
}


%Geometrie----------------------------------------------------------------------------------------------------------

\geometry{a4paper, top=25mm, left=15mm, right=15mm, bottom=25mm,headsep=10mm, footskip=10mm}
\pagestyle{fancy}
\setlength{\parindent}{0pt} %Zeileneinrückung

\fancyhf{} %Setzt voreingestellte Kopf-und Fußzeilen-Eigenschaften zurück

\lhead{\nouppercase{\leftmark}}
\chead{}
\rhead{\thepage}

\lfoot{}
\cfoot{}
\rfoot{}

\title{\vspace{0cm}{\Huge Fortgeschrittenen-Praktikum I:\\ \vspace{1cm} Kurze Halbwertszeiten}}
\author{Saskia Bondza\\Simon Stephan}
\date{Durchgeführt am 02.09.2016 und 05.09.2016}

\pretitle{%
  \begin{center}
  \LARGE
  \includegraphics[width=6cm,]{figures/siegel}\\[\bigskipamount]
}
\posttitle{\end{center}}

%neue Commands----------------------------------------------------------------------------------------------------------
\newcommand{\nab}{\vec{\nabla}} %direkter Befehl mit Vektorpfeil
\newcommand{\gra}[3][0.7]{
	\begin{minipage}[h!]{\textwidth}
		\centering
		\includegraphics[width=#1\textwidth]{figures/#2.png}
		\captionof{figure}{#3}
	\end{minipage}
	}


%Titel,Inhalt----------------------------------------------------------------------------------------------------------

\begin{document}
\pagenumbering{gobble} %verstecke Seitenzahl
\maketitle
\newpage

\thispagestyle{empty}
\tableofcontents
\newpage

%Schreiben----------------------------------------------------------------------------------------------------------
\pagenumbering{arabic} %verstecke Seitenzahl
\section{Einleitung}


In diesem Versuch soll die Halbwertszeit des $14,4$ keV-Zustands von $^{57}$Fe bestimmt werden. Dieser Zustand kann durch den Zerfall von $^{57}$Co durch Elektroneneinfang und anschließenden $\gamma$-Zerfall des $^{57}$Fe erzeugt werden. Zum Messen der Halbwertszeit dieses Zerfalls wird die Zeit zwischen der Detektierung des $122,1$ keV-Photons, welches beim Entstehen des $14,4$ keV-Zustands ausgesendet wird, und der Detektierung des $14,4$ keV-Photons, welches beim Zerfall des $14,4$ keV-Zustands in den Grundzustand ensteht, gemessen. Da nur ein kleiner Teil der mit $14,4$ keV angeregten Eisenkerne unter Aussendung eines Photons zerfällt und so nicht zu jedem $122,1$ keV-Photon ein $14,4$ keV-Photon entsteht, wird zum Messen die Methode der verzögerten Koinzidenzen benutzt.
\\\\
Zunächst werden die Energie-Spektren von $^{57}$Co und $^{241}$Am aufgenommen. Anschließend wird eine Zeiteichung des Time to Amplitude Converters (TAC) durchgeführt. Dann wird eine Untergrundmessung ohne Signalverzögerung durchgeführt um die zufälligen Koinzidenzen zu messen. Nun werden die verzögerten Koinzidenzen gemessen und daraus die Halbwertszeit des $14,4$ keV-Zustands von $^{57}$Fe bestimmt.

\vskip 50pt
\gra{Cobaldt-Zerfall}{Zerfallsschema von $^{57}$Co $^{\cite{anleitung}}$}
%\newpage
%\section[Aufgabenstellung]{Aufgabenstellung}%$^{\cite{anleitung}}$}
%Zum Bestimmen der Halbwertszeit werden zunächst die Energie-Spektren von $^{}57$Co und $^{241}$Am aufgenommen. Anschließend wird eine Zeiteichung des Time to Amplitude Converters (TAC) durchgeführt. Dann wird eine Untergrundmessung ohne Signalverzögerung durchgeführt um die zufälligen Koinzidenzen zu messen. Nun werden die verzögerten Koinzidenzen gemessen und daraus die Halbwertszeit des $14,4$ keV-Zustands von $^{57}$Fe bestimmt.







\newpage
\section{Theoretische Grundlagen}
\subsection{Zerfallsgesetz}\label{zerfallsgesetz}
Der radioaktive Zerfall ist ein statistischer Prozess d.h. der genaue Zeitpunkt zu dem ein Kern zerfällt kann nicht vorhergesagt werden, stattdessen wird mit Wahrscheinlichkeiten gearbeitet. Die Zerfallsrate $dN/dt$ ist abhängig von der momentanen ANzahl $N$ der Atome.
\[\frac{dN}{dt}=-\lambda N\]
wobei $\lambda$ die Zerfallskonstante ist, welche den Bruchteil der zu Beginn vorhandenen Kerne, die pro Zeiteinheit zerfallen beschreibt. Zerfallskonstanten einzelner Zerfälle addieren sich, zerfällt ein Kern auf mehrere Arten. Durch Lösen dieser Differentialgleichung erhält man die Anzahl der Kerne $N$ zum Zeitpunkt $t$.
\[N(t)=N_0e^{-\lambda t}\hspace{0.5cm}\]
wobei $N_0=N(t=0)$ die anfängliche Anzahl der Kerne beschreibt. Hieraus lässt sich dann die mittlere Lebensdauer $\tau$ bestimmen:
\[\tau=\frac{1}{\lambda}\]
welche mit der Halbwertszeit $T_{1/2}$ wie folgt zusammenhängt:
\[T_{1/2} = \ln2\cdot \tau = \frac{\ln2}{\lambda}\]
Eine weitere wichtige Größe ist die Aktivität $A$ welche die Anzahl der zerfälle pro Sekunde beschreibt und in Becquerel ( $Bq=1/s$) gemeesen wird:
\[A = \lambda N = \frac{N}{\tau} = \frac{N\ln2}{T_{1/2}}\]
\subsection{Radioaktive Zerfälle}
Instabile Atomkerne gehen, je nach Art des Zerfalls, unter Aussendung von ionisierender Strahlung und Aussendung von Teilchen spontan in einen anderen Atomkern über. Im Folgenden werden die verschiedenen Arten radioaktiver Zerfälle erläutert.
\subsubsection{$\alpha$-Zerfall}
Beim $\alpha$-Zerfall geht ein schwerer, instabiler Kern unter Aussendung eines $\alpha$-Teilchens, einem Heliumkern (zwei Protonen und zwei Neutronen) in einen stabilen Kern über. Die Massenzahl des neuen Atomkerns ist dabei um vier gesunken, die Ordnungszahl verringert sich um zwei.
\subsubsection{$\beta$-Zerfall}
Der $\beta$-Zerfall wird durch verschiedene Prozesse der schwachen Wechselwirkung beschrieben bei denen Elektronen  und Positronen auftreten. Bei diesem Vorgang wird durch W-Boson-Austausch ein Proton in ein Neutron umgewandelt bzw. ein Neutron in ein Proton.

\paragraph*{$\beta^-$-Zerfall:} 
Bei dieser Art des Zerfalls wandelt sich ein Neutron im Kern unter Aussendung eines Elektrons und eines Elektron-Antineutrinos in ein Proton um. 
\begin{align*}
n &\rightarrow p + e^- + \bar{\nu_e}\\
{}_Z^A X &\rightarrow {}_{Z+1}^A Y + e^- + \bar{\nu_e}
\end{align*}

Hierbei entspricht X dem Mutternuklid und Y dem Tochternuklid. Die Ordnungszahl erhöht sich bei diesem Vorgang um eine Einheit, die Massenzahl ändert sich nicht.

\paragraph*{$\beta^+$-Zerfall:}
Der $\beta^+$-Zerfall ist dem $\beta^-$-Zerfall sehr ähnlich, Hier wandelt sich ein Proton im Kern unter Aussendung eines Positrons und eines Elektron-Neutrinos in ein Neutron um.
\begin{align*}
p &\rightarrow n + e^+ + \nu_e\\
{}_Z^A X &\rightarrow {}_{Z-1}^A Y + e^+ + \nu_e
\end{align*}

Bei diesem Prozess sinkt die Kernladungszahl um eine Einheit während die Massenzahl wie auch beim $\beta^-$-Zerfall unverändert bleibt.
\subsubsection{Elektroneneinfang}
Die meisten schweren, instabilen Kerne mit Protonenüberschuss wandeln sich durch Elektroneneinfang in stabile Kerne um.
Der Elektroneneinfang führt effektiv im Kern zum gleichen Resultat wie der $\beta^+$-Zerfall, d.h. die Ordnungszahl verringert sich um eine Einheit, die Massenzahl bleibt unverändert. Meist wird aus der Kern-nächsten Schale, der K-Schale, ein Bahnelektron unter Aussendung eines Neutrinos eingefangen. Die Lücke, die hierbei entsteht, wird meist durch ein aus der L-Schale stammendes Elektron unter Emission von Röntgenstrahlung oder Auger-Elektronen (siehe \ref{auger}) gefüllt. Dieser Vorgang wiederholt sich mit den weiter außen liegenden Schalen.
\[{}_Z^A X + e^- \rightarrow\ {}_{Z-1}^A Y + \nu_e\]
\subsubsection{$\gamma$-Zerfall}

Der $\gamma$-Zerfall ist oft eine Begleiterscheinung der bereits diskutierten Kernzerfälle: Befindet sich das Tochternuklid nach dem Zerfall in einem angeregten Zustand, geht es über Aussendung eines $\gamma$-Quants (Photon)  in einen energetisch niedrigeren Zustand über. Die Energie dieser Photonen liegt dabei im Bereich von keV bis MeV. In Materie fällt die Intensität$I_d$ der $\gamma$-Strahlung exponentiell mit der Strecke $d$ ab:
\[I_d=e^{\mu d}\]
Dabei ist $\mu$ der Absorptionskoeffizent des Materials und $I_0$ die Anfangsintensität.

\subsubsection{Innere Konversion}
Konkurrierend zum $\gamma$-Zerfall gibt es auch die Innere Konversion. Bei dieser gehen angeregte Kernzustände strahlungslos in den Grundzustand über, wobei die Energie an ein Elektron übertragen und dieses abgestrahlt wird. Die so entstandene Lücke in der Atomhülle wird dann unter Emission von Röntgenstrahlung oder Auger-Elektronen (siehe \ref{auger}) gefüllt.

\subsection{Prozesse in der Atomhülle}\label{auger}
Bei manchen Zerfallsprozessen entsteht in der Atomhülle durch Emission eines Elektrons ein Loch. Dieses Loch wird gefüllt, indem ein Elektron einer höheren Schale in die Schale des Loches wechselt und dabei Energie freigibt. Diese Energie kann entweder über direkte Emission eines Photons (Röntgenstrahlung) oder über die Emission eines anderen Elektrons (Auger-Elektron) abgegeben werden. Dadurch entstehen eine oder zwei neue Lücken, welche wiederum über die weitere Emission von Röntgenstrahlung oder eines Auger-Elektrons geschlossen werden können.

\subsection{Wechselwirkung von $\gamma$-Strahlung mit Materie}
Um $\gamma$-Strahlung detektieren zu können, muss die $\gamma$-Strahlung mit Materie wechselwirken. Wir unterscheiden dabei grundsätzlich drei Prozesse: Den Photoeffekt, den Compton-Effekt und die Paarbildung.
 \subsubsection{Photo-Effekt}
 Beim Photoeffekt dringt ein Photon in das Atom ein und überträgt seine gesamte Energie an ein Elektron der inneren Schalen. Dabei wird Energie auf dieses Elektron übertragen, es wird aus der Atomhülle befreit und erhält kinetische Energie. Die hier entstandene Lücke wird über Abstrahlung eines $\gamma$-Quants oder eines Auger-Elektrons wieder gefüllt.
 Der Photoeffekt findet typischerweise bei Energien bis zu $200$ keV statt.
 \subsubsection{Compton-Effekt}
 Beim Compton-Effekt trifft ein einfallendes $\gamma$-Quant auf ein freies oder nur leicht gebundenes Elektron und überträgt einen Teil seiner Energie auf dieses. Dieser Prozess findet meist bei Energien zwischen $200$ keV und $5$ MeV statt.
 \subsubsection{Paarbildung}
 Bei der Paarbildung entsteht durch die Wechselwirkung des $\gamma$-Quants mit dem elektromagnetischen Feld des Atomkerns oder eines Elektrons ein Teilchen-Antiteilchen-Paar i.e. Elektronen-Positronen-Paar.Paarbildung ist für Energien über $1,022$ MeV möglich. Die über diesen Grenzwert hinausgehende Energie wird auf die entstandenen Teilchen übertragen, der Imuls wird vom Kern aufgenommen. Da das Positron nicht lange alleine existieren kann, vereinigt es sich unter Abstrahlung von zwei $\gamma$-Quanten mit einem Elektron.
 \subsection{Messprinzip}
 \subsubsection{Messung der Zerfallsdauern}
 
 \paragraph{Verzögerte Koinzidenzen}Um die Halbwertszeit des $14,4$ keV-Zustands von $^{57}$Fe zu bestimmen, messen wir die einzelnen Lebensdauern der angeregten $^{57}$Fe-Atome. Der Zustand entsteht unter Emission eines $122,1$ keV-$\gamma$-Quants und zerfällt unter Emission eines $14,4$ keV-$\gamma$-Quants, so dass wir die Zeitmessung mit Detektion des $122,1$ keV-Photons starten und mit Detektion des $14,4$ keV-Photons beenden könnten. Bei diesem Verfahren allerdings tritt das Problem auf, dass nur ca. 10\% der $^{57}$Fe-Atome über $\gamma$-Zerfall und ca. 90\% über innere Konversion zerfallen. Wenn nun ein $122,1$ keV-Photon detektiert wird und kein dazugehöriges $14,4$ keV-Photon erzeugt wird, dann wird die Messung nicht beendet, wodurch die Totzeit vergrößert und die Messung verfälscht wird.
 Um dieses Problem zu umgehen, benutzt man die Methode der verzögerten Koinzidenzen. Bei dieser Methode wird das Signal der $122,1$ keV-Photonen verzögert, so dass das Signal nach dem Signal der $14,4$ keV-Photonen ankommt. Dann wird das Signal der $14,4$ keV-Photonen als Start- und das der $122,1$ keV-Photonen als Stop-Signal verwendet. Die gemessene Differenz wird nun von der Verzögerung abgezogen, um die tatsächliche Lebensdauer zu erhalten.
 Der radioaktive Zerfall ist ein statistischer Prozess (siehe \ref{zerfallsgesetz}), mit der Anzahl $N(t)$ der nach einer Zeit $t$ noch vorhandenen Kerne: $N(t)=N_0\exp({-\frac t \tau})$ Die gemessenen Zeiten $\Delta t$ folgen ebenso dieser Exponentialverteilung um die mittlere Lebensdauer $\tau$.
 Durch das Vertauschen von Start- und Stoppsignal wird diese Exponentialverteilung an der y-Achse gespiegelt, sodass erst durch Zeitverzögerung die Exponentialverteilung im positiven x-Achsen-Bereich  (und damit messbar) ist. Beim Messen der verzögerten Koinzidenzen erwarten wir demnach einen exponentiellen Anstieg.
 
 \paragraph{Zufällige Koinzedenzen} Da das Start- und Stoppsignal am TAC nicht zwingend vom selben Kern, d.h. vom selben Zerfallsprozess, stammt müssen diese zufälligen Koinzidenzen, die einen Untergrund bei der Messung der verzögerten Koinzidenzen darstellen, ebenfalls gemessen werden. Diese Messung kann gleichzeitig mit der Messung der verzögerten Koinzidenzen durchgeführt werden, in dem der Delay (die Zeitverzögerung) so gewählt wird, dass der exponentielle Anstieg nur die erste Hälfte des Fensters einnimmt, sodass die zufälligen Koinzidenzen im letzten Viertel des Plots beobachtbar sind. 
 
 \subsubsection{Nachweis der $\gamma$-Strahlung mithilfe des Szintillationszählers}
 Die emittierten Photonen müssen nun detektiert und ihre Energie bestimmt werden. Dazu wird ein Energie-sensitiver Detektor, eine Kombination aus Szintillator und Photomultiplier, verwendet.
 \paragraph{Szintillator} Ein Szintillator detektiert Teilchen eines bestimmten Energiebereichs. Es gibt organische und anorganische Szintillatoren, wobei in diesem Versuch anorganische NaI(Tl)-Szintillatoren verwendet werden. Dieser besteht aus einem mit Thallium dotierten NaI-Kristall, in welchem die eintreffenden Photonen ihre Energie durch den Photo- oder Compton-Effekt an Elektronen abgeben. Je höher die Energie der Photonen ist, desto mehr Elektronen werden erzeugt. Diese Elektronen werden nun angeregt und später unter Emission  niederenergetischer Photonen wieder abgeregt. Die Dotierung mit Thallium verhindert, dass die emittierten Elektronen wieder absorbiert werden.
 \paragraph{Photomultiplier}
 Das Licht wird nun vom Szintillator über Lichtleiter zum Photomultiplier geleitet. Dieser wandelt die Lichtimpulse des Szintillators durch den Photoeffekt in elektrische Impulse um, welche proportional zur Lichtintensität sind und verstärkt diese durch Elektronenvervielfachung.
 

 
\newpage
\section{Versuchsaufbau- und durchführung}
\subsection{Geräte}

\paragraph{Detektor} Der Szintillator und der Photomultiplier bilden zusammen den Detektor.
\paragraph{Main Amplifier (MA)}
Der Hauptverstärker  verstärkt das Spannungssignal rauscharm und erzeugt einen möglichst kurzen Puls, dessen Dauer amplituden-unabhängig ist. Die Pulshöhe ist dabei annähernd proportional zur deponierten Ladungsmenge im Szintillator. Der Hauptverstärker hat zwei verschiedene Ausgänge die entweder ein bipolares Signal (zeitsensible Messungen) oder ein unipolares Signal (Aufnahme der Spektren) liefern.
\paragraph{Multi Channel Analyser}
Der Multi Channel Analyser ordnet jeden eingehenden Puls, abhängig von der Pulshöhe (also von der im Szintillator deponierten Ladungsmenge ) einem Channel zu. Das so erhaltene Energiespektrum kann als Histogramm dargestellt werden.
\paragraph{Single Channel Analyser}
Ein Single Channel Analyser selektiert aus eingehenden Signalen in dem nur für Energien in einem einstellbaren Energiefenster ein Ausgangspuls erzeugt wird.  Für die Messung der verzögerten und zufälligen Koinzedenzen sollen hier die Energiefenster dem $122$ keV peak und dem $14,4$ keV peak angepasst werden (siehe \ref{Energiefenster})
\paragraph{Time to Amplitude Converter (TAC)} Der Time to Amplitude Converter erzeugt einen in der Höhe  zur Zeitdifferenz zwischen Start- und Stoppsignal proportionalen Rechteckpuls. Ab Eingang des Startsignals wird hierzu ein Kondensator durch Anlegen eines konstanten Stroms aufgeladen. Das Stoppsignal stoppt den Aufladevorgang und wandelt den aufintegrierten Strom in ein Spannungssignal mit fester zeitlicher Länge um. Je nach Amplitude dieses Pulses wird das Signal nun auf die entsprechenden Chanel des MCA verteilt. Um also später bei der Messung der verzögerten und zufälligen Koinzedenzen den einzelnen Channles Zeitdifferenzen zuordnen zu können muss eine Zeiteichung bei gleicher Channelanzahl durchgeführt werden. (siehe \ref{zeiteichung})

\vskip 30 pt
\gra[0.5]{TAC}{Funktionsweise des TAC}



\subsection{Aufnahme der Energiespektren}
Zunächst werden die Energiespektren von Kobalt und Americium aufgenommen. Dabei muss das Spektrum jeweils von beiden Detektoren aufgenommen werden und zusätzlich beim Kobalt auch in beiden Orientierungen. So soll bestimmt werden, mit welchem Detektor nachher welcher Peak ($14,4$ keV bzw.$122$ keV) gemessen wird. Hierfür verwenden wir folgende Schaltung:
\vskip 30 pt
\gra{Schaltung_1}{Versuchsaufbau 1: Aufnahme der Energiespektren}

Das Spektrum von $^{241}$Am dient zur Energiekalibrierung. Hierzu werden die Peaks des Americiumspektrums den einzelnen Zerfällen des Zerfallsschemas zugeordnet und so der Kalibrierungsfaktor zwischen der Energie und der Channelanzahl bestimmt. Dies dient dann zur Identifizierung der $^{57}$Fe-Peaks. Außerdem soll an Hand der Energiespektren der optimale Versuchsaufbau bestimmt werden, da z.B. die Erkennbarkeit des $14,4$ keV peak sehr verschieden ist.
Zur Aufnahme der Energiespektren verwendeten wir folgende Einstellungen:




\subsection{Setzen der Energiefenster}\label{Energiefenster}

Im zweiten Teil des Versuchs sollen die Energiefenster an den Single Channel Analyzern eingestellt werden. Hierzu verwenden wir folgende Schaltung:
\vskip 30 pt
\gra{Schaltung_2}{Versuchsaufbau 2: Einstellen der Energiefenster}


Anhand des Spektrums von  $^{241}$Am identifizieren wir die $122$ keV und $14,4$ keV peaks (siehe \ref{Auswertung}) und setzen die Energiefenster der SCAs so, dass nur für den $122$ keV peak bzw. den $14,4$ keV peak ein Signal erzeugt wird.
\subsection{Messung der verzögerten und zufälligen Koinzidenzen}
Zur Messung der verzögerten und zufälligen Koinzedenzen werden die Einstellungen aus den vorherigen Teilen des Experiments übernommen. Wir verwenden hier folgende Schaltung:

\vskip 30 pt
\gra{Schaltung_3}{Versuchsaufbau 3: Messen der verzögerten und zufälligen Koinzidenzen}


Die Delay-Einstellung wird so gewählt, dass ca. in der ersten Hälfte der exponentielle Anstieg zu sehen ist, im dritten Viertel der ausgeschmierte Abfall zu sehen ist und im letzten Viertel die zufälligen Koinzidenzen, d.h. der Untergrund zu sehen ist.
 
\subsection{Zeitkalibrierung des TAC} \label{zeiteichung}

\vskip 30 pt
\gra{Schaltung_4}{Versuchsaufbau 4: Zeitkalibrierung des TAC\label{zeitkalibrierung}} 


Zur Messung der Zeitkalibrierung des TAC wird das Signal eines SCA aufgeteilt und ein Zweig unverzögert als Startsignal an den TAC weitergeleitet während der zweite Zweig über die Delay-Box verzögert wird und als Stoppsignal an den TAC angelegt wird. Wir erhalten so durch Einstellen verschiedener Delay-Zeiten Chanel-Zeit-Paare, mit denen die Zeitkalibrierung durchgeführt wird.
Zu beachten ist hierbei, dass die gleiche Chanel-Anzahl wie beim vorherigen Versuchsteil gewählt wird.
Der Versuchsaufbau zu dieser Messung ist in Abbildung \ref{zeitkalibrierung} dargestellt.


\newpage
\section{Auswertung}

Wir haben für die Auswertung des Versuchs die Programmiersprache R verwendet.

\subsection{Energiespektren}

\paragraph{Energiekalibration} Wir haben zunächst die Peaks des $^{241}$Am - Spektrums für beide Detektoren mit Hilfe von Gauß-fits bestimmt.

\vskip 30 pt
\gra[0.8]{Americium}{$^{241}$Am - Spektren mit Gauß-Fit}


Für die Einzelnen Fits erhalten wir dabei folgende Daten:
\vskip 10 pt

\begin{table}[h!]

\centering
\begin{tabular}{l|c|c}
&linker Szintillator&rechter Szintillator\\\hline
Peak 1&

 $\mu    = 31.8\pm0.4$&$\mu    = 29.61\pm0.19$\\
&$ \sigma = 8.3\pm0.31$&$\sigma = 7.6\pm0.2$\\\hline


Peak 2&

 $\mu    = 40.75\pm0.18$&$\mu    = 44.1\pm0.2$\\
 &$\sigma = 8.22\pm0.18$&$\sigma = 7.19\pm0.18$\\\hline


Peak 3&

 $\mu    = 93.518\pm0.017$&$\mu    = 99.23\pm0.03$\\
 &$\sigma = 8.523\pm0.016$&$\sigma = 7.10\pm0.02$
\end{tabular}
\caption{Peaks des $^{241}$Am-Spektrums}
\end{table}

\vskip 20 pt
Diese Peaks werden nun an Hand des Zerfallsschemas von $^{241}$Am (Abbildung 8) den entprechenden Energien zugeordnet:

\vskip 30 pt
\gra[0.5]{Zerfall_Americium}{Zerfallsschema von $^{241}$Am }


Die Peaks sind damit den Energien $26.3$ keV, $33.2$ keV und $59.5$ keV zuzuordnen. Wir können nun für beide Szintillatoren die Energiekalibrierung durchführen. Dazu führen wir für die erhaltenen Energie-Chanel-Paare einen Linearen Fit durch:

\vskip 30 pt
\gra[0.8]{linker_szinti}{Energiekalibrierung Linker Szintillator}
\vskip 20 pt

\gra[0.8]{rechter_Szinti}{Energiekalibrierung Rechter Szintillator}


Wir habe hier die zuvor berechneten Standardabweichung der gefitteten Gaußverteilungen sinnvollerweise als Fehler angegeben.
Hierbei ist anzumerken, das ein linearer Fit mit drei Punkten statistisch wenig sinnvoll ist. Die gefundenen Peaks im Kobalt-Spektrum werden wir daher mit der Channelzahl für die jeweiligen Peaks und dem Energie-Literaturwert in den Fit eintragen und überprüfen, ob sie im Rahmen des Fehlers auf der Geraden liegen.
Für die aufgenommenen Kobaltspektren haben wir die Peaks ebenfalls Gauß gefittet:

\vskip 30 pt
\gra[0.9]{Cobalt}{Energiespektren von Kobalt}


In Abbildung 11 sieht man, dass der $14.4$ keV Peak nur in einem Graphen gut zu erkennen ist, nämlich beim rechten Szintillator in Orientierung 1. Wir haben daher auch nur für diesen Graph einen Gauß-Fit für den $14.4$ keV Peak durchgeführt. 

\vskip 30 pt

\begin{table}[h!]

\begin{tabular}{l|c|c}
\centering
&links&rechts\\\hline
Orientierung 1&
 $\mu    = 188.34\pm0.13$&Peak 1: $\mu    = 21.7\pm0.4$\\
 &$\sigma = 15.40\pm0.12$&$\sigma = 6.3\pm1.3$\\
 &&Peak 2: $\mu    = 201.75\pm0.16$\\
  &&$\sigma = 12.05\pm0.15$\\\hline
Orientierung 2&
 $\mu    = 190.8\pm0.2$& $\mu    = 196.9\pm0.2$\\
 &$\sigma = 15.55\pm0.19$&$\sigma = 12.98\pm0.19$
 \end{tabular}
 \caption{Peaks des $^{57}$Co-Spektrums }
\end{table}

\vskip 30 pt

Wir tragen nun die Kobaltpaks mit Literatur-Energiewert($122$ keV bzw. $14.4$ keV) und ihren Standardabweichungen in die Linearen Fits ein, um wie bereits erwähnt, ein Gefühl für die Güte dieser Fits zu bekommen.

\vskip 30 pt
\gra[0.8]{Linker_Szinti_1}{Energiekalibrierung des linken Szintillators mit Kobaltpeaks}


\vskip 30 pt
\gra[0.8]{Rechter_Szinti_1}{Energiekalibrierung des linken Szintillators mit Kobaltpeaks}


Anhand von Abbildung 13 stellen wir fest, dass die Kobaltpeaks im Rahmen von zwei Standardabweichungen alle auf der Geraden liegen. Wir haben hier die die Fehler des Linearen Fits berücksichtigt, indem wir die steilste und flachste Ausgleichsgerqade in blau eingezeichnet haben. Für den linken Szintillator stimmen die Kobabltpeaks sogar innerhalb einer Standardabweichung mit der Geraden überein. Die größere Abweichung beim rechten Szintillator lässt sich durch die sehr kleinen Fehler der linearen Regression erklären, die bei drei Werten eher zufällig enstanden sein können und nicht den realen Fehlern entsprechen. Hierfür spricht auch, dass die einzelnen Punkte zunächst ähnliche Abweichungen wie beim linken Szintillator zeigen. Da die Energiekalibrierung nur für das Einstellen der Energiefenster wichtig ist, ist sie daher vollkommen ausreichend.

\subsection{Hauptmessung}
\newpage
\section{Zusammenfassung/Diskussion}


\newpage
\section{Anhang}

\subsection{Tabellen}

%\subsubsection{$\alpha$-Plateau Samarium}
%\lstinputlisting[language=MATLAB]{Rohdaten/alphaPlateau_Sm.txt}


%\newpage
%\subsection{Quellcode (MATLAB)}
%\lstinputlisting[language=MATLAB]{Rohdaten/alpha.m}

\newpage
\subsection{Laborheft}
%\begin{minipage}{\textwidth}
%\centering
%\includegraphics[width=0.9\textwidth]{figures/IMG_20151002_141014.jpg}
%\end{minipage}

\newpage
\listoffigures

%Literatur----------------------------------------------------------------------------------------------------------

%\cite{les}
\newpage
\thispagestyle{empty}
\begin{thebibliography}{9}

\bibitem{anleitung}
 (Zerfall von $^{57}$Co): http://hacol13.physik.uni-freiburg.de/fp/Versuche/FP1/FP1-6-KurzeHalbwertzeiten/Anleitung.pdf
  

  
%\bibitem{molmasse}
%  \emph{http://www.convertunits.com/molarmass/<ELEMENTNAME AUF ENGLISCH>}, Stand 28.09.2015
  

\end{thebibliography}

\end{document}